\chapter{Einführung}
\section{Zustands- und Prozessgrößen}

\textbf{Zustand}\\
= beschreibt eine Situation für den Betrachtungsgegenstand X zu einer bestimmten Zeit y und den gegebenen Rahmenbedingungen z
\begin{center}
	$\text{Zustand} = \text{Situation}(x,yz)$
\end{center}
Zustände können:
\begin{itemize}
	\item  labil
	\item stabil
	\item veränderlich sein.
\end{itemize}
\begin{center}
	Faktoren \& Umgebungsbedingungen $\rightarrow$ Zustände $\rightarrow$ phys./chem. Größen für Beschreibung
\end{center}

\textbf{Zustandsgrößen}\\
= physik./chem. Größen um momentanen Zustand eines Systems zu beschreiben
\begin{itemize}
	\item wegunabhängig, d.h. "`nur Ergebnis zählt"'
	\item können aber gegenseitig von einander abhängen, z.B. $p,V,T$
	\item Summe der Teilchen repräsentiert
\end{itemize}

\textbf{Prozess}\\
= Veränderung eines Systems und seiner Umgebung beim Übergang vom Zustand 1 in Zustand 2
\begin{itemize}
	\item reversibel oder irreversibel
	\item spontan oder nach Zwang/Aktivierung
\end{itemize}

\textbf{Prozessgröße}\\
= physik./chem. Größen, die den Übergang des Systems x vom Zustand 1 in Zustand 2 beschreiben
$\rightarrow$ wegabhängig, z.B. Wärme $Q$, Arbeit $W$\\

\newpage

\textbf{Extensive Größen A}
\begin{itemize}
	\item nicht normiert
	\item verdoppelt sich, wenn sich System verdoppelt
	\item schlecht vergleichbar
	\item Bsp.: Volumen 2 Tassen Kaffee
\end{itemize}

\textbf{Intensive Größen a}
\begin{itemize}
	\item normiert
	\item verdoppelt sich \textit{nicht}, wenn sich System verdoppelt
	\item gut vergleichbar
	\item Bsp.: Dichte des Kaffees
\end{itemize}